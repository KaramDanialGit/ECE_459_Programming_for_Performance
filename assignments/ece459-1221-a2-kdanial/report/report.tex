\documentclass[12pt]{article}

\usepackage[letterpaper, hmargin=0.75in, vmargin=0.75in]{geometry}
\usepackage{float}
\usepackage[T1]{fontenc}

\pagestyle{empty}

\title{ECE 459: Programming for Performance\\Assignment 2}
\author{Karam Danial}
\date{\today}

\begin{document}

\maketitle

\section{Message Passing}
\subsection{Logic}
\hspace{4mm} Since a single-threaded implementation of JWT verification is very inefficient, consider implementing the
solver with multiple threads and message passing using channels to communicate between all threads. Intuitively, a set
of threads can be called at different initial states (i.e. a different starting character from a partitioned set of chars
from DEFAULT\_ALPHABETS) and attempt to produce the token. The threads can communicate with each other via message-passing
through two channels: one channel sending and receiving to each thread and one communicating with the main thread to
receive the answer. The first channel will notify other threads to terminate in case of a solution, and the second channel
will pass the answer outside the thread spawn handler.

\subsection{Implementation}
\hspace{4mm} First, determine the number of threads to run by calling num\_cpus::get() as recommended in the lab manual.
This will spawn one thread per available CPU. Second, I partition DEFAULT\_ALPHABETS by chunks according to the number of
threads via the chunks() function. This way, each thread gets a set of unique characters to traverse and check; ultimately,
dividing the characters amongst our threads will decrease expedite the check\_all process. Now that the code assigns a set
of characters for each thread to iterate through and verify, we must implement the message-passing functionality for threads
to notify each other in case of a solution - or otherwise return None. Two channels are set up to communicate (1) amongst
threads and (2) to the main thread. Thread\_sender is used to send a string "Answer Found!" in case check\_all returns an
answer. Thread\_receiver is then used by each thread to check if the channel is empty or not. If the channel is empty,
threads continue to work for a solution. In contrast, a non-empty channel tells an on-going thread that we have found a
solution, please terminate. Similarly, when check\_all finds a solution, main\_sender passes its solution, which is later
received after each thread handler is joined and printed to the console. Obviously, if the main\_receiver still contains
None, "No answer found" is printed to the console.

\hspace{4mm} Note, calling an implementation of JwtSolver in multi-threaded solution, a new instance of JwtSolver is
passed into each thread spawn. Therefore, we assign the cloning attribute to the solver struct by inserting
"#[derive(Clone)]" above the struct's definition.

\section{Shared Memory}
\subsection{Logic}
\hspace{4mm} The overall logic for implementing multi-threaded JWT validation with shared memory is very similar to message-
passing. We still use the same number of threads as available CPUs and the same alphabet-partitioning system. Thread spawning
is also identical to that of the previous implementation. However, instead of using channels, a buffer declared outside the
for loop is declared and appropriately controlled by a MutexGuard.

\subsection{Implementation}
\hspace{4mm} The newly declared buffer of type Arc<Mutex<Option<Vec<u8>\hspace{0.25mm}>\hspace{0.25mm}>\hspace{0.25mm}>
is declared and used by each thread to write a solution or check the presence of a solution. With each iteration, access
to this buffer is locked by a MutexGuard. This mutex ensures only one thread is within the critical section and interacting
with the buffer at a time. Similar to message-passing, if one thread generates a valid JWT, the result is written into the
buffer. At the beginning of each character state (i.e. the for loop iterating through character chunks), the thread checks
if the buffer is empty. If the buffer is empty, the thread continues to find a solution. Otherwise, the thread terminates,
and the cycle continues until all other threads finish executing or recognize a solution and terminate. To access the buffers
contents, Arc::try\_unwrap(buffer).unwrap() and into\_inner().unwrap(). Of course, a match case is used to determine if in
fact we did get a valid answer to display. Naturally, "No answer found" is printed in the absence of a non-None result.

\end{document}

